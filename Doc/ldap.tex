% $Id$

% ==== 1. ====
% The section prologue.  Give the section a title and provide some
% meta-information.  References to the module should use
% \refbimodindex, \refstmodindex, \refexmodindex or \refmodindex, as
% appropriate.

\section{\module{ldap} --- LDAP library interface module}

\declaremodule{extension}{ldap}		% not standard, in C

\platform{UNIX}

% Author of the module code;
\moduleauthor{Michael Str\"oder}{python-ldap-dev@lists.sourceforge.net}
% Author of the documentation,
\sectionauthor{Michael Str\"oder}{michael@stroeder.com}

% Leave at least one blank line after this, to simplify ad-hoc tools
% that are sometimes used to massage these files.
\modulesynopsis{Access to an underlying LDAP C library.}

% ==== 2. ====
% Give a short overview of what the module does.
% If it is platform specific, mention this.
% Mention other important restrictions or general operating principles.

This module provides access to the LDAP
(Lightweight Directory Access Protocol) \C\ API implented
in OpenLDAP 2.
It is similar to the C API, with the notable differences
that lists are manipulated via Python
list operations and errors appear as exceptions.

For far more detailed information on the \C\ interface, 
please see the (expired) draft-ietf-ldapext-ldap-c-api-04.

This documentation is current for the Python LDAP module, version
$\version$.
Source and binaries are available from
\url{http://python-ldap.sourceforge.net/}.

% ==== 3. ====
% List the public functions defined by the module.  Begin with a
% standard phrase.  You may also list the exceptions and other data
% items defined in the module, insofar as they are important for the
% user.

\subsection{Functions}

The \module{ldap} module defines the following functions:

% ---- 3.1. ----
% For each function, use a ``funcdesc'' block.  This has exactly two
% parameters (each parameters is contained in a set of curly braces):
% the first parameter is the function name (this automatically
% generates an index entry); the second parameter is the function's
% argument list.  If there are no arguments, use an empty pair of
% curly braces.  If there is more than one argument, separate the
% arguments with backslash-comma.  Optional parts of the parameter
% list are contained in \optional{...} (this generates a set of square
% brackets around its parameter).  Arguments are automatically set in
% italics in the parameter list.  Each argument should be mentioned at
% least once in the description; each usage (even inside \code{...})
% should be enclosed in \var{...}.

\begin{funcdesc}{initialize}{uri} % -> LDAPObject
  Opens a new connection with an LDAP server, and return an LDAP object
  (see \ref{ldap-objects}) used to perform operations on that server.
  Parameter \var{uri} has to be a valid LDAP URL.
  \begin{seealso}
  \seerfc{2255}{The LDAP URL Format}{}
  \end{seealso}
\end{funcdesc}

\begin{funcdesc}{open}{host \optional{, port=\constant{PORT}}} % -> LDAPObject
  Opens a new connection with an LDAP server, and return an LDAP object
  (see \ref{ldap-objects}) used to perform operations on that server.
  \var{host} is a string containing solely the host name. \var{port}
  is an integer specifying the port where the LDAP server is
  listening (default is 389).
  Note: Using this function is deprecated.
\end{funcdesc}

\begin{funcdesc}{dn2ufn}{dn} % -> string
  Turns \var{dn} into a more user-friendly form, stripping off type names.
  \begin{seealso}
  \seerfc{1781}{Using the Directory to Achieve User Friendly Naming}{}
  \end{seealso}
\end{funcdesc}

\begin{funcdesc}{explode_dn}{dn \optional{, notypes=\constant{0}}} % -> list
  This function takes \var{dn} and breaks it up into its component parts. 
  Each part is known as an RDN (Relative Distinguished Name). The
  \var{notypes} parameter is used to specify that only the RDN values be 
  returned and not their types.
  For example, the DN \code{"cn=Bob, c=US"} would be
  returned as either \code{["cn=Bob", "c=US"]} or \code{["Bob","US"]}
  depending on whether \var{notypes} was \constant{0} or \constant{1},
  respectively.
  \begin{seealso}
  \seerfc{2253}{Lightweight Directory Access Protocol (v3):
           UTF-8 String Representation of Distinguished Names}{}
  \end{seealso}
\end{funcdesc}

\begin{funcdesc}{explode_rdn}{rdn \optional{, notypes=\constant{0}}} % -> list
  This function takes a (multi-valued) \var{rdn} and breaks it up
  into a list of characteristic attributes. The
  \var{notypes} parameter is used to specify that only the RDN values be 
  returned and not their types.
\end{funcdesc}

\begin{funcdesc}{is_ldap_url}{url} % -> boolean
  This function returns true if \var{url} `looks like' an LDAP URL 
  (as opposed to some other kind of URL). 
  \begin{seealso}
    \seerfc{2255}{The LDAP URL Format}{}
  \end{seealso}
\end{funcdesc}

\begin{funcdesc}{get_option}{option} % -> None
  This function returns the value of the global option
  specified by \var{option}.
\end{funcdesc}

\begin{funcdesc}{set_option}{option, invalue} % -> None
  This function sets the value of the global option
  specified by \var{option} to \var{invalue}.
\end{funcdesc}


% ---- 3.2. ----
% Data items are described using a ``datadesc'' block.  This has only
% one parameter: the item's name.

\subsection{Constants}

The module defines various constants.

\subsubsection{General}

\begin{datadesc}{PORT}
  The assigned TCP port number (389) that LDAP servers listen on.
\end{datadesc}

\subsubsection{Options}

For use with functions and method set_option() and get_option() the
following option identifiers are defined as constants:

\begin{datadesc}{OPT_API_FEATURE_INFO}
\end{datadesc}

\begin{datadesc}{OPT_API_INFO}
\end{datadesc}

\begin{datadesc}{OPT_CLIENT_CONTROLS}
\end{datadesc}

\begin{datadesc}{OPT_DEBUG_LEVEL}
\end{datadesc}

\begin{datadesc}{OPT_DEREF}
\end{datadesc}

\begin{datadesc}{OPT_ERROR_STRING}
\end{datadesc}

\begin{datadesc}{OPT_HOST_NAME}
\end{datadesc}

\begin{datadesc}{OPT_MATCHED_DN}
\end{datadesc}

\begin{datadesc}{OPT_NETWORK_TIMEOUT}
\end{datadesc}

\begin{datadesc}{OPT_PRIVATE_EXTENSION_BASE}
\end{datadesc}

\begin{datadesc}{OPT_PROTOCOL_VERSION}
\end{datadesc}

\begin{datadesc}{OPT_REFERRALS}
\end{datadesc}

\begin{datadesc}{OPT_REFHOPLIMIT}
\end{datadesc}

\begin{datadesc}{OPT_RESTART}
\end{datadesc}

\begin{datadesc}{OPT_SERVER_CONTROLS}
\end{datadesc}

\begin{datadesc}{OPT_SIZELIMIT}
\end{datadesc}

\begin{datadesc}{OPT_SUCCESS}
\end{datadesc}

\begin{datadesc}{OPT_TIMELIMIT}
\end{datadesc}

\begin{datadesc}{OPT_TIMEOUT}
\end{datadesc}

\begin{datadesc}{OPT_URI}
\end{datadesc}

\begin{datadesc}{OPT_X_SASL_AUTHCID}
\end{datadesc}

\begin{datadesc}{OPT_X_SASL_AUTHZID}
\end{datadesc}

\begin{datadesc}{OPT_X_SASL_MECH}
\end{datadesc}

\begin{datadesc}{OPT_X_SASL_REALM}
\end{datadesc}

\begin{datadesc}{OPT_X_SASL_SECPROPS}
\end{datadesc}

\begin{datadesc}{OPT_X_SASL_SSF}
\end{datadesc}

\begin{datadesc}{OPT_X_SASL_SSF_EXTERNAL}
\end{datadesc}

\begin{datadesc}{OPT_X_SASL_SSF_MAX}
\end{datadesc}

\begin{datadesc}{OPT_X_SASL_SSF_MIN}
\end{datadesc}

\begin{datadesc}{OPT_X_TLS}
\end{datadesc}

\begin{datadesc}{OPT_X_TLS_ALLOW}
\end{datadesc}

\begin{datadesc}{OPT_X_TLS_CACERTDIR}
\end{datadesc}

\begin{datadesc}{OPT_X_TLS_CACERTFILE}
\end{datadesc}

\begin{datadesc}{OPT_X_TLS_CERTFILE}
\end{datadesc}

\begin{datadesc}{OPT_X_TLS_CIPHER_SUITE}
\end{datadesc}

\begin{datadesc}{OPT_X_TLS_CTX}
\end{datadesc}

\begin{datadesc}{OPT_X_TLS_DEMAND}
\end{datadesc}

\begin{datadesc}{OPT_X_TLS_HARD}
\end{datadesc}

\begin{datadesc}{OPT_X_TLS_KEYFILE}
\end{datadesc}

\begin{datadesc}{OPT_X_TLS_NEVER}
\end{datadesc}

\begin{datadesc}{OPT_X_TLS_RANDOM_FILE}
\end{datadesc}

\begin{datadesc}{OPT_X_TLS_REQUIRE_CERT}
\end{datadesc}

\begin{datadesc}{OPT_X_TLS_TRY}
\end{datadesc}


% --- 3.3. ---
% Exceptions are described using a ``excdesc'' block.  This has only
% one parameter: the exception name.  Exceptions defined as classes in
% the source code should be documented using this environment, but
% constructor parameters must be ommitted.

\subsection{Exceptions}
\label{subsec:exceptfrommeth}

The module defines the following exceptions:

\begin{excdesc}{LDAPError}
This is the base class of all execeptions raised by the module \module{ldap}.
Unlike the \C\ interface, errors are not returned as result codes, but
are instead turned into exceptions, raised as soon an the error condition 
is detected.

The exceptions are accompanied by a dictionary possibly
containing an string value for the key \constant{'desc'} 
(giving an English description of the error class)
and/or a string value for the key \constant{'info'}
(giving a string containing more information that the server may have sent).

A third possible field of this dictionary is \constant{'matched'} and
is set to a truncated form of the name provided or alias dereferenced
for the lowest entry (object or alias) that was matched.

\end{excdesc}

\begin{excdesc}{ADMINLIMIT_EXCEEDED}

\end{excdesc}
\begin{excdesc}{AFFECTS_MULTIPLE_DSAS}

\end{excdesc}
\begin{excdesc}{ALIAS_DEREF_PROBLEM}
A problem was encountered when dereferencing an alias.
(Sets the \constant{'matched'} field.)
\end{excdesc}
\begin{excdesc}{ALIAS_PROBLEM}
An alias in the directory points to a nonexistent entry.
(Sets the \constant{'matched'} field.)
\end{excdesc}
\begin{excdesc}{ALREADY_EXISTS}
The entry already exists. E.g. the \var{dn} specified with \method{add()}
already exists in the DIT.
\end{excdesc}
\begin{excdesc}{}

\end{excdesc}
\begin{excdesc}{AUTH_UNKNOWN}
The authentication method specified to \method{bind()} is not known.
\end{excdesc}
\begin{excdesc}{BUSY}
The DSA is busy.
\end{excdesc}
\begin{excdesc}{CLIENT_LOOP}

\end{excdesc}
\begin{excdesc}{COMPARE_FALSE}
A compare operation returned false.
(This exception should never be seen because \method{compare()} returns
a boolean result.)
\end{excdesc}
\begin{excdesc}{COMPARE_TRUE}
A compare operation returned true.
(This exception should never be seen because \method{compare()} returns
a boolean result.)
\end{excdesc}
\begin{excdesc}{CONFIDENTIALITY_REQUIRED}
Indicates that the session is not protected by a protocol such
as Transport Layer Security (TLS), which provides session
confidentiality.
\end{excdesc}
\begin{excdesc}{CONNECT_ERROR}

\end{excdesc}
\begin{excdesc}{CONSTRAINT_VIOLATION}
An attribute value specified or an operation started violates some
server-side constraint
(e.g., a postalAddress has too many lines or a line that is too long
or a password is expired).
\end{excdesc}
\begin{excdesc}{CONTROL_NOT_FOUND}

\end{excdesc}
\begin{excdesc}{DECODING_ERROR}
An error was encountered decoding a result from the LDAP server.
\end{excdesc}
\begin{excdesc}{ENCODING_ERROR}
An error was encountered encoding parameters to send to the LDAP server.
\end{excdesc}
\begin{excdesc}{FILTER_ERROR}
An invalid filter was supplied to method{search()}
(e.g. unbalanced parentheses).
\end{excdesc}
\begin{excdesc}{INAPPROPRIATE_AUTH}
Inappropriate authentication was specified (e.g. \constant{LDAP_AUTH_SIMPLE}
was specified and the entry does not have a userPassword attribute).
\end{excdesc}
\begin{excdesc}{INAPPROPRIATE_MATCHING}
Filter type not supported for the specified attribute.
\end{excdesc}
\begin{excdesc}{INSUFFICIENT_ACCESS}
The user has insufficient access to perform the operation.
\end{excdesc}
\begin{excdesc}{INVALID_CREDENTIALS}
Invalid credentials were presented during \method{bind()} or
\method{simple_bind()}.
(e.g., the wrong password).
\end{excdesc}
\begin{excdesc}{INVALID_DN_SYNTAX}
A syntactically invalid DN was specified. (Sets the \constant{'matched'} field.)
\end{excdesc}
\begin{excdesc}{INVALID_SYNTAX}
An attribute value specified by the client did not comply to the
syntax defined in the server-side schema.
\end{excdesc}
\begin{excdesc}{IS_LEAF}
The object specified is a leaf of the diretcory tree.
Sets the \constant{'matched'} field of the exception dictionary value.
\end{excdesc}
\begin{excdesc}{LOCAL_ERROR}
Some local error occurred. This is usually due to failed memory allocation.
\end{excdesc}
\begin{excdesc}{LOOP_DETECT}
A loop was detected.
\end{excdesc}
\begin{excdesc}{MORE_RESULTS_TO_RETURN}

\end{excdesc}
\begin{excdesc}{NAMING_VIOLATION}
A naming violation occurred. This is raised e.g. if the LDAP server
has constraints about the tree naming.
\end{excdesc}
\begin{excdesc}{NO_OBJECT_CLASS_MODS}
Modifying the objectClass attribute as requested is not allowed
(e.g. modifying structural object class of existing entry).
\end{excdesc}
\begin{excdesc}{NOT_ALLOWED_ON_NONLEAF}
The operation is not allowed on a non-leaf object.
\end{excdesc}
\begin{excdesc}{NOT_ALLOWED_ON_RDN}
The operation is not allowed on an RDN.
\end{excdesc}
\begin{excdesc}{NOT_SUPPORTED}

\end{excdesc}
\begin{excdesc}{NO_MEMORY}

\end{excdesc}
\begin{excdesc}{NO_OBJECT_CLASS_MODS}
Object class modifications are not allowed.
\end{excdesc}
\begin{excdesc}{NO_RESULTS_RETURNED}

\end{excdesc}
\begin{excdesc}{NO_SUCH_ATTRIBUTE}
The attribute type specified does not exist in the entry.
\end{excdesc}
\begin{excdesc}{NO_SUCH_OBJECT}
The specified object does not exist in the directory.
Sets the \constant{'matched'} field of the exception dictionary value.
\end{excdesc}
\begin{excdesc}{OBJECT_CLASS_VIOLATION}
An object class violation occurred when the LDAP server checked
the data sent by the client against the server-side schema
(e.g. a "must" attribute was missing in the entry data).
\end{excdesc}
\begin{excdesc}{OPERATIONS_ERROR}
An operations error occurred.
\end{excdesc}
\begin{excdesc}{OTHER}
An unclassified error occurred.
\end{excdesc}
\begin{excdesc}{PARAM_ERROR}
An ldap routine was called with a bad parameter.
\end{excdesc}
\begin{excdesc}{PARTIAL_RESULTS}
Partial results only returned. This exception is raised if
a referral is received when using LDAPv2.
(This exception should never be seen with LDAPv3.)
\end{excdesc}
\begin{excdesc}{PROTOCOL_ERROR}
A violation of the LDAP protocol was detected.
\end{excdesc}
\begin{excdesc}{RESULTS_TOO_LARGE}
The result does not fit into a UDP packet. This happens only when using
UDP-based CLDAP (connection-less LDAP) which is not supported anyway.
\end{excdesc}
\begin{excdesc}{SASL_BIND_IN_PROGRESS}

\end{excdesc}
\begin{excdesc}{SERVER_DOWN}
The  LDAP  library  can't  contact the LDAP server.
\end{excdesc}
\begin{excdesc}{SIZELIMIT_EXCEEDED}
An LDAP size limit was exceeded.
This could be due to a `sizelimit' configuration on the LDAP server.
\end{excdesc}
\begin{excdesc}{STRONG_AUTH_NOT_SUPPORTED}
The LDAP server does not support strong authentication.
\end{excdesc}
\begin{excdesc}{STRONG_AUTH_REQUIRED}
Strong authentication is required  for the operation.
\end{excdesc}
\begin{excdesc}{TIMELIMIT_EXCEEDED}
An LDAP time limit was exceeded.
\end{excdesc}
\begin{excdesc}{TIMEOUT}
A timelimit was exceeded while waiting for a result from the server.
\end{excdesc}
\begin{excdesc}{TYPE_OR_VALUE_EXISTS}
An  attribute  type or attribute value specified already 
exists in the entry.
\end{excdesc}
\begin{excdesc}{UNAVAILABLE}
The DSA is unavailable.
\end{excdesc}
\begin{excdesc}{UNAVAILABLE_CRITICAL_EXTENSION}
Indicates that the LDAP server was unable to satisfy a request
because one or more critical extensions were not available. Either
the server does not support the control or the control is not appropriate
for the operation type.
\end{excdesc}
\begin{excdesc}{UNDEFINED_TYPE}
An attribute type used is not defined in the server-side schema.
\end{excdesc}
\begin{excdesc}{UNWILLING_TO_PERFORM}
The  DSA  is  unwilling to perform the operation.
\end{excdesc}
\begin{excdesc}{USER_CANCELLED}
The operation was cancelled via the \method{abandon()} method.
\end{excdesc}

The above exceptions are raised when a result code from an underlying API
call does not indicate success.

% ---- 3.4. ----
% Other standard environments:
%
%  classdesc	- Python classes; same arguments are funcdesc
%  methoddesc	- methods, like funcdesc but has an optional parameter 
%		  to give the type name: \begin{methoddesc}[mytype]{name}{args}
%		  By default, the type name will be the name of the
%		  last class defined using classdesc.  The type name
%		  is required if the type is implemented in C (because 
%		  there's no classdesc) or if the class isn't directly 
%		  documented (if it's private).
%  memberdesc	- data members, like datadesc, but with an optional
%		  type name like methoddesc.

\subsection{LDAP Objects \label{ldap-objects}}

% This label is generally useful for referencing this section, but is
% also used to give a filename when generating HTML.

%\noindent
LDAP objects are returned by \function{open()} and \function{initialize()}.
The connection is automatically unbound and closed 
when the LDAP object is deleted.

Most methods on LDAP objects initiate an asynchronous request to the LDAP server 
and return a message id that can be used later to retrieve the result
with \method{result()}.
Methods with names ending in `\constant{_s}' are the synchronous form 
and wait for and return with the server's result, or with
\constant{None} if no data is expected.

LDAP objects, have the following methods:

%%------------------------------------------------------------
%% fileno

\begin{methoddesc}{fileno}{}
Return the file descriptor associated with the connection, for use
with the \module{select} module.
(This method is available only if the underlying library can supply it.)
\end{methoddesc}

\subsubsection{LDAP operations}

%%------------------------------------------------------------
%% abandon

\begin{methoddesc}[LDAP]{abandon}{msgid}
Abandons or cancels an LDAP operation in progress. The \var{msgid} argument
should be the message ID of an outstanding LDAP operation as returned by
the asynchronous methods \method{search()}, \method{modify()}, etc. 
The caller can expect that the result
of an abandoned operation will not be returned from a future call to 
\method{result()}.
\end{methoddesc}

%%------------------------------------------------------------
%% add

\begin{methoddesc}[LDAP]{add}{dn, modlist} % -> int
     \methodline{add_s}{dn, modlist}
Performs an LDAP add operation. The \var{dn} argument is the distinguished
name (DN) of the entry to add, and \var{modlist} is a list of attributes to be
added. The modlist is similar the one passed to \method{modify()}, except that the
operation integer is omitted from the tuples in modlist. You might want to
look into sub-module l\refmodule{ldap.modlist} for generating the modlist.
\end{methoddesc}

%%------------------------------------------------------------
%% bind

\begin{methoddesc}[LDAP]{bind}{who, cred, method} % -> int
     \methodline[LDAP]{bind_s}{who, cred, method} % -> None
     \methodline[LDAP]{simple_bind}{who, passwd} % -> None
     \methodline[LDAP]{simple_bind_s}{who, passwd} % -> None
After an LDAP object is created, and before any other operations can be
attempted over the connection, a bind operation must be performed.

This method attempts to bind with the LDAP server using 
either simple authentication, or Kerberos (if available).
The first and most general method, \method{bind()},
takes a third parameter, \var{method}, which can currently solely
be \constant{AUTH_SIMPLE}.
\end{methoddesc}

%%------------------------------------------------------------
%% compare

\begin{methoddesc}[LDAP]{compare}{dn, attr, value} % -> int
     \methodline[LDAP]{compare_s}{dn, attr, value} % -> tuple
Perform an LDAP comparison between the attribute named \var{attr} of 
entry \var{dn}, and the value \var{value}. The synchronous form
returns \constant{0} for false, or \constant{1} for true.
The asynchronous form returns the message ID of the initiated request, 
and the result of the asynchronous compare can be obtained using 
\method{result()}.  

Note that the asynchronous technique yields the answer
by raising the exception objects \constant{COMPARE_TRUE} or
\constant{COMPARE_FALSE}.

\textbf{Note} A design fault in the LDAP API prevents \var{value} 
from containing nul characters.
\end{methoddesc}

%%------------------------------------------------------------
%% delete

\begin{methoddesc}[LDAP]{delete}{dn} % -> int
     \methodline[LDAP]{delete_s}{dn} % -> None
Performs an LDAP delete operation on \var{dn}. The asynchronous form
returns the message id of the initiated request, and the result can be obtained
from a subsequent call to \method{result()}.
\end{methoddesc}

%%------------------------------------------------------------
%% modify

\begin{methoddesc}[LDAP]{modify}{ dn, modlist } % -> int
     \methodline[LDAP]{modify_s}{ dn, modlist } % -> None
Performs an LDAP modify operation on an entry's attributes. 
The \var{dn} argument is the distinguished name (DN) of the entry to modify,
and \var{modlist} is a list of modifications to make to that entry.

Each element in the list \var{modlist} should be a tuple of the form 
\code{(mod_op,mod_type,mod_vals)},
where \var{mod_op} indicates the operation (one of \constant{MOD_ADD}, 
\constant{MOD_DELETE}, or \constant{MOD_REPLACE}),
\var{mod_type} is a string indicating the attribute type name, and 
\var{mod_vals} is either a string value or a list of string values to add, 
delete or replace respectively.  For the delete operation, \var{mod_vals}
may be \constant{None} indicating that all attributes are to be deleted.

The asynchronous method \method{modify()} returns the message ID of the 
initiated request.

You might want to like into sub-module \refmodule{ldap.modlist} for
generating \var{modlist}.
\end{methoddesc}

%%------------------------------------------------------------
%% modrdn

\begin{methoddesc}[LDAP]{modrdn}{dn, newrdn \optional{, delold=\constant{1}}}
		%-> int
     \methodline[LDAP]{modrdn_s}{dn, newrdn \optional{, delold=\constant{1}}}
		% -> None
Perform a `modify RDN' operation, (i.e. a renaming operation).
These routines take \var{dn} (the DN
of the entry whose RDN is to be changed, and \var{newrdn}, the new RDN to
give to the entry. The optional parameter \var{delold} is used to specify
whether the old RDN should be kept as an attribute of the entry or not.
(Note: This may not actually be supported by the underlying library.)
The asynchronous version returns the initiated message id.
\end{methoddesc}

%%------------------------------------------------------------
%% rename

\begin{methoddesc}[LDAP]{rename}{dn, newrdn \optional{, newsuperior=\constant{None}} \optional{, delold=\constant{1}}}
		%-> int
     \methodline[LDAP]{rename_s}{dn, newrdn \optional{, newsuperior=\constant{None} \optional{, delold=\constant{1}}}}
		% -> None
Perform a `Rename' operation, (i.e. a renaming operation).
These routines take \var{dn} (the DN
of the entry whose RDN is to be changed, and \var{newrdn}, the new RDN to
give to the entry.
The optional parameter \var{newsuperior} is used to specify
a new parent DN for moving an entry in the tree
(not all LDAP servers support this).
The optional parameter \var{delold} is used to specify
whether the old RDN should be kept as an attribute of the entry or not.
\end{methoddesc}

%%------------------------------------------------------------
%% result

\begin{methoddesc}[LDAP]{result}{\optional{ msgid=\constant{RES_ANY} \optional{, all=\constant{1} \optional{, timeout=\constant{-1}}}}} % -> tuple
This method is used to wait for and return the result of an operation
previously initiated by one of the LDAP \emph{asynchronous} operations
(eg \method{search()}, \method{modify()}, etc.) 

The \var{msgid} parameter is the integer identifier returned by that
method. 
The identifier is guaranteed to be unique across an LDAP session,
and tells the \method{result()} method to request the result of that
specific operation.
If a result is desired from any one of the in-progress operations,
\var{msgid} should be specified as the constant \constant{RES_ANY}.

The \var{all} parameter only has meaning for \method{search()} responses
and is used to select whether a single entry of the search
response should be returned, or to wait for all the results
of the search before returning.

A search response is made up of zero or more search entries
followed by a search result. If \var{all} is 0, search entries will
be returned one at a time as they come in, via separate calls
to \method{result()}. If all is 1, the search response will be returned
in its entirety, i.e. after all entries and the final search
result have been received.

For \var{all} set to 0, result tuples
trickle in (with the same message id), and with the result type
\constant{'RES_SEARCH_ENTRY'}, until the final result which has a result
type of \constant{'RES_SEARCH_RESULT'} and a (usually) empty data field.
When all is set to 1, only one result is returned, with a
result type of RES_SEARCH_RESULT, and all the result tuples
listed in the data field.

The \var{timeout} parameter is a limit on the number of seconds that the
method will wait for a response from the server. 
If \var{timeout} is negative (which is the default),
the method will wait indefinitely for a response.
The timeout can be expressed as a floating-point value, and
a value of \constant{0} effects a poll.
If a timeout does occur, a \exception{TIMEOUT} exception is raised,
unless polling, in which case \constant{(None, None)} is returned.

The \method{result()} method returns a tuple of the form 
\code{(\textit{result-type}, \textit{result-data})}.
The first element, \textit{result-type} is a string, being one of:
\constant{'RES_BIND'}, \constant{'RES_SEARCH_ENTRY'},
\constant{'RES_SEARCH_REFERENCE'}, \constant{'RES_SEARCH_RESULT'}, 
\constant{'RES_MODIFY'}, \constant{'RES_ADD'}, \constant{'RES_DELETE'}, 
\constant{'RES_MODRDN'}, or \constant{'RES_COMPARE'}.
(The module constants \constant{RES_*} are set to these strings,
for your convenience.)

If \var{all} is \constant{0}, one response at a time is returned on
each call to \method{result()}, with termination indicated by 
\textit{result-data} being an empty list.

See \method{search()} for a description of the search result's 
\var{result-data}, otherwise the \var{result-data} is normally meaningless.

\end{methoddesc}

%%------------------------------------------------------------
%% search

\begin{methoddesc}[LDAP]{search}{base, scope \optional{,filterstr=\constant{'(objectClass=*)'} \optional{, 
	attrlist=\constant{None} \optional{, attrsonly=\constant{0}}}}} %->int
     \methodline[LDAP]{search_s}{base, scope \optional{,filterstr=\constant{'(objectClass=*)'} \optional{, 
     	attrlist=\constant{None} \optional{, attrsonly=\constant{0}}}}} 
		%->list|None
    \methodline[LDAP]{search_st}{base, scope \optional{,filterstr=\constant{'(objectClass=*)'} \optional{,
    	attrlist=\constant{None} \optional{, attrsonly=\constant{0} 
	\optional{, timeout=\constant{-1}}}}}}
    \methodline[LDAP]{search_ext}{
        base, scope
        \optional{,filterstr=\constant{'(objectClass=*)'}
        \optional{, attrlist=\constant{None} 
        \optional{, attrsonly=\constant{0} 
        \optional{, serverctrls=\constant{None} 
        \optional{, clientctrls=\constant{None} 
	\optional{, timeout=\constant{-1} 
        \optional{, sizelimit=\constant{0}}}}}}}}}
		%->int
    \methodline[LDAP]{search_ext_s}{
        base, scope
        \optional{,filterstr=\constant{'(objectClass=*)'}
        \optional{, attrlist=\constant{None} 
        \optional{, attrsonly=\constant{0} 
        \optional{, serverctrls=\constant{None} 
        \optional{, clientctrls=\constant{None} 
	\optional{, timeout=\constant{-1} 
        \optional{, sizelimit=\constant{0}}}}}}}}}
		%->list|None
Perform an LDAP search operation, with \var{base} as the DN of the entry
at which to start the search, \var{scope} being one of 
\constant{SCOPE_BASE} (to search the object itself), 
\constant{SCOPE_ONELEVEL} (to search the object's immediate children), or
\constant{SCOPE_SUBTREE} (to search the object and all its descendants).

The
\var{filterstr} argument is a string representation of the filter to apply in
the search.

\begin{seealso}
\seerfc{2254}{The String Representation of LDAP Search Filters}{}
\end{seealso}

Each result tuple is of the form \code{(\var{dn},\var{attrs})}, 
where \var{dn} is a string containing the DN (distinguished name) of the
entry, and \var{attrs} is a dictionary containing the attributes associated
with the entry. The keys of \var{attrs} are strings, and the associated
values are lists of strings.

The DN in \var{dn} is extracted using the underlying \cfunction{ldap_get_dn()}
function,
which may raise an exception if the DN is malformed.

If \var{attrsonly} is non-zero, the values of \var{attrs} will be meaningless
(they are not transmitted in the result).

The retrieved attributes can be limited with the \var{attrlist} parameter.
If \var{attrlist} is \constant{None}, all the attributes of each entry are returned.

To do: \var{serverctrls}

To do: \var{clientctrls}

The synchronous form with timeout, \method{search_st()} or \method{search_ext_s()},
will block for at most \var{timeout} seconds (or indefinitely if \var{timeout}
is negative). A \exception{TIMEOUT} exception is raised if no result is received
within the specified time.

The amount of search results retrieved can be limited with the
\var{sizelimit} parameter when using \method{search_ext()}
or \method{search_ext_s()} (client-side search limit). If non-zero
not more than \var{sizelimit} results are returned by the server.

\end{methoddesc}

%%------------------------------------------------------------
%% set_rebind_proc

\begin{methoddesc}{set_rebind_proc}{func}
If a referral is returned from the server, automatic
re-binding can be achieved by providing a function that accepts as an argument
the newly opened LDAP object and returns the tuple \code{(who, cred, method)}.

Passing a value of \constant{None} for \var{func} will disable
this facility. 

Because of restrictions in the implementation, only one
rebinding function is supported at any one time.
In addition, this method is only
available if support is available in the underlying library (LDAP_REFERRALS).
\end{methoddesc}

%%------------------------------------------------------------
%% unbind

\begin{methoddesc}[int]{unbind}{}
       \methodline{unbind_s}{}
This call is used to unbind from the directory, terminate the current
association, and free resources. Once called, the connection to the
LDAP server is closed and the LDAP object is marked invalid.
Further invocation of methods on the object will yield exceptions.

The \method{unbind()} and \method{unbind_s()} methods are both
synchronous in nature
\end{methoddesc}

\subsubsection{LDAP options}

\begin{methoddesc}[LDAP]{get_option}{option} % -> None
  This function returns the value of the LDAPObject option
  specified by \var{option}.
\end{methoddesc}

\begin{methoddesc}[LDAP]{set_option}{option, invalue} % -> None
  This function sets the value of the LDAPObject option
  specified by \var{option} to \var{invalue}.
\end{methoddesc}

%%------------------------------------------------------------
%% manage_dsa_it

\begin{methoddesc}[LDAP]{manage_dsa_it}{enable, \optional{, critical=\constant{0}}}
		%-> None
Enables or disables manageDSAit mode (see draft-zeilenga-ldap-namedref)
according to the specified integer flag \var{enable}. The
integer flag \var{critical} specifies if the use of this extended
control is marked critical.

\textbf{Note}
This method is somewhat immature and might vanish in future versions
if full support for extended controls will be implemented. You have been
warned!
\end{methoddesc}

%%============================================================
%% attributes

\subsubsection{Object attributes}

If the underlying library provides enough information,
each LDAP object will also have the following attributes.
These attributes are mutable unless described as read-only.

%%------------------------------------------------------------
%% deref

\begin{memberdesc}[LDAP]{deref} % -> int
    Controls for when an automatic dereference of a referral occurs.
    This must be one of
    \constant{DEREF_NEVER}, \constant{DEREF_SEARCHING}, \constant{DEREF_FINDING},
    or \constant{DEREF_ALWAYS}.
\end{memberdesc}

%%------------------------------------------------------------
%% errno

\begin{memberdesc}[LDAP]{errno} % -> int
   \memberline[LDAP]{error} % -> string
   \memberline[LDAP]{matched} % -> string
    These attributes are set after an exception has been raised, and
    are also included with the value raised. 
    (See also \ref{subsec:exceptfrommeth}.)
    (All read-only.)
\end{memberdesc}

%%------------------------------------------------------------
%% lberoptions

\begin{memberdesc}[LDAP]{lberoptions} % -> int
    Options for the BER library.
\end{memberdesc}

%%------------------------------------------------------------
%% options

\begin{memberdesc}[LDAP]{options} % -> int
    General options. This field is the bitiwse OR of the flags
	\constant{OPT_REFERRALS} (follow referrals), and
	\constant{OPT_RESTART}   (restart the \cfunction{select()} system call
			      when interrupted).
\end{memberdesc}

%%------------------------------------------------------------
%% refhoplimit

\begin{memberdesc}[LDAP]{refhoplimit} % -> int
    Maximum number of referrals to follow before raising an exception.
    Defaults to 5.
\end{memberdesc}

%%------------------------------------------------------------
%% sizelimit

\begin{memberdesc}[LDAP]{sizelimit} % -> int
    Limit on size of message to receive from server. 
    Defaults to \constant{NO_LIMIT}.
\end{memberdesc}

%%------------------------------------------------------------
%% timelimit

\begin{memberdesc}[LDAP]{timelimit} % -> int
    Limit on waiting for any response, in seconds. 
    Defaults to \constant{NO_LIMIT}.
\end{memberdesc}

%%------------------------------------------------------------
%% valid

\begin{memberdesc}[LDAP]{valid} % -> int
    If zero, the connection has been unbound. See \method{unbind()} for
    more information.
    (read-only)
\end{memberdesc}


% ==== 4. ====
% Now is probably a good time for a complete example.  (Alternatively,
% an example giving the flavor of the module may be given before the
% detailed list of functions.)

\subsection{Example \label{ldap-example}}

The following example demonstrates how to open an LDAP server using the
\module{ldap} module.

\begin{verbatim}
>>> import ldap
>>> l = ldap.initialize("ldap://my-ldap-server.my-domain:389")
>>> l.simple_bind_s("","")
>>> l.search_s("o=My Organisation, c=AU", ldap.SCOPE_SUBTREE, "objectclass=*")
\end{verbatim}

% ==== 5. ====
% If your module defines new object types (for a built-in module) or
% classes (for a module written in Python), you should list the
% methods and instance variables (if any) of each type or class in a
% separate subsection.

