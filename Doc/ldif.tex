% $Id$

% ==== 1. ====
% The section prologue.  Give the section a title and provide some
% meta-information.  References to the module should use
% \refbimodindex, \refstmodindex, \refexmodindex or \refmodindex, as
% appropriate.

\section{\module{ldif} ---
         LDIF parser and generator}

\declaremodule{standard}{ldif}

% Author of the module code;
\moduleauthor{python-ldap developers}{python-ldap-dev@lists.sourceforge.net}
% Author of the documentation,
\sectionauthor{Michael Str\"oder}{michael@stroeder.com}

% Leave at least one blank line after this, to simplify ad-hoc tools
% that are sometimes used to massage these files.
\modulesynopsis{Parses and generates LDIF files}


% ==== 2. ====
% Give a short overview of what the module does.
% If it is platform specific, mention this.
% Mention other important restrictions or general operating principles.

This module parses and generates LDAP data in the format LDIF.

\begin{seealso}
\seerfc{2849}{The LDAP Data Interchange Format (LDIF) - Technical Specification}{}
\end{seealso}

\subsection{Example \label{ldif-example}}

The following example demonstrates how to parse an LDIF file
with \module{ldif} module.

\begin{verbatim}
To do...
\end{verbatim}

The following example demonstrates how to write LDIF output
of an LDAP entry with \module{ldif} module.

\begin{verbatim}
>>> import sys,ldif
>>> entry={'objectClass':['top','person'],'cn':['Michael Stroeder'],'sn':['Stroeder']}
>>> dn='cn=Michael Stroeder,ou=Test'
>>> ldif_writer=ldif.LDIFWriter(sys.stdout)
>>> ldif_writer.unparse(dn,entry)
dn: cn=Michael Stroeder,ou=Test
cn: Michael Stroeder
objectClass: top
objectClass: person
sn: Stroeder

>>> 
\end{verbatim}
