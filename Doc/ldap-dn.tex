% $Id$

\section{\module{ldap.dn} ---
  LDAP Distinguished Name handling }

\declaremodule{standard}{ldap.dn}

% Author of the module code;
\moduleauthor{python-ldap project}{python-ldap-dev@lists.sourceforge.net}

\modulesynopsis{LDAP Distinguished Name handling.}

\begin{seealso}
For LDAPv3 DN syntax see:
\seerfc{4514}{Lightweight Directory Access Protocol (LDAP): String Representation of Distinguished Names}{}
For LDAPv2 DN syntax (obsoleted by LDAPv3) see:
\seerfc{1779}{A String Representation of Distinguished Names}{}
\end{seealso}

The \module{ldap.dn} module defines the following functions:

\begin{funcdesc}{escape_dn_chars}{s} % -> string
  This function escapes characters in string \var{s} which
  are special in LDAP distinguished names. You should use
  this function when building LDAP DN strings from arbitrary input.
\end{funcdesc}

\begin{funcdesc}{str2dn}{s \optional{, flags=\constant{0}}} % -> list
  This function takes \var{s} and breaks it up into its component parts
  down to AVA level. The optional parameter \var{flags} describes the DN format of s
  (see \ref{ldap-dn-flags}).
\end{funcdesc}


\begin{funcdesc}{dn2str}{dn} % -> string
  This function takes a decomposed DN in \var{dn} and returns
  a single string. It's the inverse to \function{str2dn()}.
  Special characters are escaped with the help of
  function \function{escape_dn_chars()}.
\end{funcdesc}


\begin{funcdesc}{explode_dn}{dn \optional{, notypes=\constant{0} \optional{, flags=\constant{0}}}} % -> list
  This function takes \var{dn} and breaks it up into its component parts. 
  Each part is known as an RDN (Relative Distinguished Name). The optional
  \var{notypes} parameter is used to specify that only the RDN values be 
  returned and not their types. The optional parameter \var{flags}
  describes the DN format of s (see \ref{ldap-dn-flags}).

  This function is emulated by function \function{str2dn()}
  since the function ldap_explode_dn() in the C library is deprecated.
\end{funcdesc}

\begin{funcdesc}{explode_rdn}{rdn \optional{, notypes=\constant{0} \optional{, flags=\constant{0}}}} % -> list
  This function takes a (multi-valued) \var{rdn} and breaks it up
  into a list of characteristic attributes. The optional
  \var{notypes} parameter is used to specify that only the RDN values be 
  returned and not their types. The optional \var{flags} parameter
  describes the DN format of s (see \ref{ldap-dn-flags}).

  This function is emulated by function \function{str2dn()}
  since the function ldap_explode_rdn() in the C library is deprecated.
\end{funcdesc}


\subsection{Examples \label{ldap-dn-example}}

Splitting a LDAPv3 DN to AVA level:
\begin{verbatim}
>>> ldap.dn.str2dn('cn=Michael Str\xc3\xb6der,dc=stroeder,dc=com',flags=ldap.DN_FORMAT_LDAPV3)
[[('cn', 'Michael Str\xc3\xb6der', 4)], [('dc', 'stroeder', 1)], [('dc', 'com', 1)]]
\end{verbatim}

Splitting a LDAPv2 DN into RDN parts:
\begin{verbatim}
>>> ldap.dn.explode_dn('cn=Michael Stroeder;dc=stroeder;dc=com',flags=ldap.DN_FORMAT_LDAPV2)
['cn=Michael Stroeder', 'dc=stroeder', 'dc=com']
\end{verbatim}

Splitting a multi-valued RDN:
>>> ldap.dn.explode_rdn('cn=Michael Stroeder+mail=michael@stroeder.com',flags=ldap.DN_FORMAT_LDAPV2)
['cn=Michael Stroeder', 'mail=michael@stroeder.com']

Splitting a LDAPv3 DN with a multi-valued RDN into its AVA parts:
\begin{verbatim}
>>> ldap.dn.str2dn('cn=Michael Stroeder+mail=michael@stroeder.com,dc=stroeder,dc=com')
[[('cn', 'Michael Stroeder', 1), ('mail', 'michael@stroeder.com', 1)], [('dc', 'stroeder', 1)], [('dc', 'com', 1)]]
\end{verbatim}
