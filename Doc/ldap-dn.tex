% $Id$

\section{\module{ldap.dn} ---
  LDAP Distinguished Name handling }

\declaremodule{standard}{ldap.dn}

% Author of the module code;
\moduleauthor{python-ldap project}{python-ldap-dev@lists.sourceforge.net}

\modulesynopsis{LDAP Distinguished Name handling.}

\begin{seealso}
\seerfc{4514}{Lightweight Directory Access Protocol (LDAP): String Representation of Distinguished Names.}{}
\end{seealso}

The \module{ldap.dn} module defines the following functions:

\begin{funcdesc}{escape_dn_chars}{s} % -> string
  This function escapes characters in string \var{s} which
  are special in LDAP distinguished names. You should use
  this function when building LDAP DN strings from arbitrary input.
\end{funcdesc}

\begin{funcdesc}{str2dn}{s \optional{, flags=\constant{0}}} % -> list
  This function takes \var{s} and breaks it up into its component parts
  down to AVA level.
  The optional parameter \var{flags} describes the DN format of s.
  \begin{seealso}
  \manpage{ldap_str2dn}{3}
  \end{seealso}
\end{funcdesc}


\begin{funcdesc}{dn2str}{dn} % -> string
  This function takes a decomposed DN in \var{dn} and returns
  a single string. It's the inverse to \function{str2dn()}.
  Special characters are escaped with the help of
  function \function{escape_dn_chars()}.
\end{funcdesc}


\begin{funcdesc}{explode_dn}{dn \optional{, notypes=\constant{0} \optional{, flags=\constant{0}}}} % -> list
  This function takes \var{dn} and breaks it up into its component parts. 
  Each part is known as an RDN (Relative Distinguished Name). The
  \var{notypes} parameter is used to specify that only the RDN values be 
  returned and not their types.
  For example, the DN \code{"cn=Bob, c=US"} would be
  returned as either \code{["cn=Bob", "c=US"]} or \code{["Bob","US"]}
  depending on whether \var{notypes} was \constant{0} or \constant{1},
  respectively.

  This function is emulated by function \function{str2dn()}
  since the function ldap_explode_dn() in the C library is deprecated.
\end{funcdesc}

\begin{funcdesc}{explode_rdn}{rdn \optional{, notypes=\constant{0} \optional{, flags=\constant{0}}}} % -> list
  This function takes a (multi-valued) \var{rdn} and breaks it up
  into a list of characteristic attributes. The
  \var{notypes} parameter is used to specify that only the RDN values be 
  returned and not their types.

  This function is emulated by function \function{str2dn()}
  since the function ldap_explode_rdn() in the C library is deprecated.
\end{funcdesc}

