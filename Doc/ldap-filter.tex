% $Id$

\section{\module{ldap.filter} ---
  LDAP filter handling }

\declaremodule{standard}{ldap.filter}

% Author of the module code;
\moduleauthor{python-ldap project}{python-ldap-dev@lists.sourceforge.net}

\modulesynopsis{LDAP filter handling.}

\begin{seealso}
\seerfc{4515}{Lightweight Directory Access Protocol (LDAP): String Representation of Search Filters.}{}
\end{seealso}

The \module{ldap.filter} module defines the following functions:

\begin{funcdesc}{escape_filter_chars}{assertion_value\optional{, escape_mode=\constant{0}}} % -> string

This function escapes characters in \var{assertion_value} which
are special in LDAP filters. You should use this function when
building LDAP filter strings from arbitrary input.

\var{escape_mode} means:
If \constant{0} only special chars mentioned in RFC 4515 are escaped.
If \constant{1} all NON-ASCII chars are escaped.
If \constant{2} all chars are escaped.

\end{funcdesc}

\begin{funcdesc}{filter_format}{filter_template, assertion_values} % -> string

This function applies \function{escape_filter_chars()} to each of the strings in
list \var{assertion_values}. After that \var{filter_template} containing
as many \constant{\%s} placeholders as count of assertion values is
used to build the whole filter string.

\end{funcdesc}
