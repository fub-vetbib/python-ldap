
% $Id$

\subsection{Functions}

The following functions are available at the module level:

%%------------------------------------------------------------
%% dn2ufn

\begin{funcdesc}{dn2ufn}{dn}
Turns the DN \var{dn} into a more user-friendly form, stripping off type names.
See RFC 1781 ``Using the Directory to Achieve User Friendly Naming''
for more details on the UFN format.
\end{funcdesc}

%%------------------------------------------------------------
%% explode_dn

\begin{funcdesc}{explode_dn}{dn \optional{\, notypes=\code{0}}}
This function takes the DN \var{dn} and breaks it up into its component parts. 
Each part is known as an RDN (Relative Distinguished Name). The
\var{notypes} parameter is used to specify that only the RDN values be returned
and not their types. For example, the DN \code{"cn=Bob, c=US"} would be
returned as either \code{["cn=Bob", "c=US"]} or \code{["Bob","US"]}
depending on whether \var{notypes} was \code{0} or \code{1}, respectively.
\end{funcdesc}

%%------------------------------------------------------------
%% is_ldap_url

\begin{funcdesc}{is_ldap_url}{url}
This function returns true if \var{url} `looks like' an LDAP URL 
(as opposed to some other kind of URL). 
\end{funcdesc}

%%------------------------------------------------------------
%% open

\begin{funcdesc}{open}{host \optional{\, port=\code{PORT}}}
Opens a new connection with an LDAP server, and returns an LDAP object
representative of this.
\end{funcdesc}
